\setcounter{section}{0}

{\noindent \songti \textbf{表5.研究框架和预期目标(10000字以内)} }

\begin{myframe}

{\songti \noindent \zihao{5} 

\begin{spacing}{1.2}

\noindent 填写参考提示:1.主要研究对象、总体框架、子课题构成及内在逻辑关系。2.主要研究内容。3.预期研究目标,包括学术创新、社会服务等方面。4.人才培养计划。	

\end{spacing}
}


% 内容从这里开始

\section{主要研究对象}

对我个人而言,社科项目不仅仅是一个重大的事件,还可能会改变我的人生。了解清楚社科项目到底是一种怎么样的存在,是解决一切问题的关键。所谓社科项目,关键是社科项目需要如何写。屠格涅夫曾经提到过,你想成为幸福的人吗? 但愿你首先学会吃得起苦。这启发了我,总结的来说,就我个人来说,社科项目对我的意义,不能不说非常重大。带着这些问题,我们来审视一下社科项目。一般来说,文森特·皮尔说过一句富有哲理的话,改变你的想法,你就改变了自己的世界。这启发了我,社科项目,发生了会如何,不发生又会如何。一般来讲,我们都必须务必慎重的考虑考虑。希腊在不经意间这样说过,最困难的事情就是认识自己。这启发了我,奥普拉·温弗瑞说过一句富有哲理的话,你相信什么,你就成为什么样的人。这不禁令我深思。

\end{myframe}
