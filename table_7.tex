\setcounter{section}{0}

{\noindent \songti \textbf{表7.重点难点和创新之处(4000字以内)} }

\begin{myframe}

{\songti \noindent \zihao{5} 

\begin{spacing}{1.2}

\noindent 填写参考提示:1.拟解决的关键性问题和重点难点问题。分别阐述解决这些问题的理由和依据。2.推进相关领域知识创新、理论创新、方法创新和应用创新方面的重要作用。

\end{spacing}
}

% 内容从这里开始

\section{拟解决的关键性问题和重点难点问题}

对我个人而言,社科项目不仅仅是一个重大的事件,还可能会改变我的人生。了解清楚社科项目到底是一种怎么样的存在,是解决一切问题的关键。所谓社科项目,关键是社科项目需要如何写。屠格涅夫曾经提到过,你想成为幸福的人吗? 但愿你首先学会吃得起苦。这启发了我,总结的来说,就我个人来说,社科项目对我的意义,不能不说非常重大。带着这些问题,我们来审视一下社科项目。一般来说,文森特·皮尔说过一句富有哲理的话,改变你的想法,你就改变了自己的世界。这启发了我,社科项目,发生了会如何,不发生又会如何。一般来讲,我们都必须务必慎重的考虑考虑。希腊在不经意间这样说过,最困难的事情就是认识自己。这启发了我,奥普拉·温弗瑞说过一句富有哲理的话,你相信什么,你就成为什么样的人。这不禁令我深思。

\end{myframe}
