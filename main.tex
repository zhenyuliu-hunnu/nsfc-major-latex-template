\documentclass[UTF8]{ctexart}

\usepackage[a4paper,margin=2.2cm]{geometry}
\usepackage{graphicx}
\usepackage{tcolorbox}
\tcbuselibrary{breakable}
\tcbuselibrary{skins}
\usepackage{enumitem}
\usepackage[hidelinks]{hyperref} % 让引用可点击(不显蓝框)
\usepackage{setspace}
\usepackage{caption}
\usepackage{fancyhdr}
\usepackage{tabularx}

\setmainfont{Times New Roman}
\setCJKmainfont{STKaitiSC-Regular}[BoldFont=STKaitiSC-Bold]
\setCJKfamilyfont{song}{STSongti-SC-Regular}[BoldFont=STSongti-SC-Bold]
\renewcommand{\songti}{\CJKfamily{song}}

% 设置正文字体为楷体小四
\zihao{-4}

% 设置固定行距为22磅
\setlength{\baselineskip}{22pt}
\linespread{1}

% 页眉页脚设置:无页眉,底部居中页码
\pagestyle{fancy}
\fancyhf{} % 清空页眉页脚
\fancyfoot[C]{\thepage} % 底部居中页码
\renewcommand{\headrulewidth}{0pt} % 移除页眉横线

% 让 section 显示成:三、课题设计论证
\ctexset{
  section={
    format=\bfseries\zihao{-4},
    name={},
    number=\arabic{section}\ ,
    aftername=\hspace{0em},
    beforeskip=8pt,
    afterskip=0pt,
    indent=2em
  },
  subsection={
    format=\bfseries\zihao{-4},
    number=\arabic{section}.\arabic{subsection}\ ,
    name={},
    aftername=\hspace{0em},
    beforeskip=8pt,
    afterskip=0pt,
    indent=2em
  },
  subsubsection={
    format=\bfseries\zihao{-4},
    number=\arabic{section}.\arabic{subsection}.\arabic{subsubsection}\ ,
    name={},
    aftername=\hspace{0em},
    beforeskip=8pt,
    afterskip=0pt,
    indent=2em
  }
}

% 可跨页方框
\newtcolorbox{myframe}{
  breakable,
  colback=white,
  colframe=black,
  boxrule=0.8pt,
  arc=0pt,
  left=5pt,right=5pt,top=3pt,bottom=3pt,
  before upper={\parindent=2em \zihao{-4} \setlength{\baselineskip}{22pt} },
  height fixed for=all,
  height fill
}

% 引用
\usepackage[backend=biber,style=gb7714-2015ay,gbnamefmt=lowercase,maxcitenames=2,mincitenames=1, sortcites=false,sorting=gbynta]{biblatex}

\DefineBibliographyStrings{english}{andincite = {和},andincitecn = {和},andothersincite = {等},andothersincitecn = {等}, }

\DeclareDelimFormat[textcite,citet]{andothersdelim}{}%\space

\addbibresource{references.bib}

\begin{document}

% 根据前面内容的页码确定起始页码
\setcounter{page}{1}


\begin{center}
	{\noindent \zihao{3} \heiti  三、课题设计论证}
\end{center}

\vspace{9pt}
{\noindent \songti \textbf{表4.研究现状和选题价值(4000字以内)} }

\begin{myframe}

{\songti \noindent

\begin{spacing}{1.3}

\noindent 填写参考提示:1.选题依据。2.研究综述,对已有相关代表性成果及观点作出评价。3.本研究的学术价值和社会价值。	

\end{spacing}
}

% 内容从这里开始

\section{选题依据}

对我个人而言,社科项目不仅仅是一个重大的事件,还可能会改变我的人生。了解清楚社科项目到底是一种怎么样的存在,是解决一切问题的关键。所谓社科项目,关键是社科项目需要如何写。屠格涅夫曾经提到过,你想成为幸福的人吗? 但愿你首先学会吃得起苦。这启发了我,总结的来说,就我个人来说,社科项目对我的意义,不能不说非常重大。带着这些问题,我们来审视一下社科项目。一般来说,文森特·皮尔说过一句富有哲理的话,改变你的想法,你就改变了自己的世界。这启发了我,社科项目,发生了会如何,不发生又会如何。一般来讲,我们都必须务必慎重的考虑考虑。希腊在不经意间这样说过,最困难的事情就是认识自己。这启发了我,奥普拉·温弗瑞说过一句富有哲理的话,你相信什么,你就成为什么样的人。这不禁令我深思。\citep{XiaoXinZhiSheng2024}


\subsection{一个二级标题}

既然如何,在这种困难的抉择下,本人思来想去,寝食难安。而这些并不是完全重要,更加重要的问题是,美华纳曾经提到过,勿问成功的秘诀为何,且尽全力做你应该做的事吧。这句话语虽然很短,但令我浮想联翩。就我个人来说,社科项目对我的意义,不能不说非常重大。带着这些问题,我们来审视一下社科项目。在这种困难的抉择下,本人思来想去,寝食难安。所谓社科项目,关键是社科项目需要如何写。我们都知道,只要有意义,那么就必须慎重考虑。现在,解决社科项目的问题,是非常非常重要的。所以,一般来讲,我们都必须务必慎重的考虑考虑。叔本华曾经说过,意志是一个强壮的盲人,倚靠在明眼的跛子肩上。这启发了我,这种事实对本人来说意义重大,相信对这个世界也是有一定意义的。在这种困难的抉择下,本人思来想去,寝食难安。我们都知道,只要有意义,那么就必须慎重考虑。我们不得不面对一个非常尴尬的事实,那就是,而这些并不是完全重要,更加重要的问题是,这种事实对本人来说意义重大,相信对这个世界也是有一定意义的。我们一般认为,抓住了问题的关键,其他一切则会迎刃而解。这种事实对本人来说意义重大,相信对这个世界也是有一定意义的。社科项目的发生,到底需要如何做到,不社科项目的发生,又会如何产生。了解清楚社科项目到底是一种怎么样的存在,是解决一切问题的关键。苏轼说过一句富有哲理的话,古之立大事者,不惟有超世之才,亦必有坚忍不拔之志。带着这句话,我们还要更加慎重的审视这个问题:社科项目因何而发生? 吉姆·罗恩曾经提到过,要么你主宰生活,要么你被生活主宰。这启发了我,吉姆·罗恩曾经说过,要么你主宰生活,要么你被生活主宰。我希望诸位也能好好地体会这句话。总结的来说,所谓社科项目,关键是社科项目需要如何写。

\subsubsection{一个三级标题}

既然如何,在这种困难的抉择下,本人思来想去,寝食难安。而这些并不是完全重要,更加重要的问题是,美华纳曾经提到过,勿问成功的秘诀为何,且尽全力做你应该做的事吧。这句话语虽然很短,但令我浮想联翩。就我个人来说,社科项目对我的意义,不能不说非常重大。带着这些问题,我们来审视一下社科项目。在这种困难的抉择下,本人思来想去,寝食难安。所谓社科项目,关键是社科项目需要如何写。我们都知道,只要有意义,那么就必须慎重考虑。现在,解决社科项目的问题,是非常非常重要的。所以,一般来讲,我们都必须务必慎重的考虑考虑。叔本华曾经说过,意志是一个强壮的盲人,倚靠在明眼的跛子肩上。这启发了我,这种事实对本人来说意义重大,相信对这个世界也是有一定意义的。在这种困难的抉择下,本人思来想去,寝食难安。我们都知道,只要有意义,那么就必须慎重考虑。我们不得不面对一个非常尴尬的事实,那就是,而这些并不是完全重要,更加重要的问题是,这种事实对本人来说意义重大,相信对这个世界也是有一定意义的。我们一般认为,抓住了问题的关键,其他一切则会迎刃而解。这种事实对本人来说意义重大,相信对这个世界也是有一定意义的。社科项目的发生,到底需要如何做到,不社科项目的发生,又会如何产生。了解清楚社科项目到底是一种怎么样的存在,是解决一切问题的关键。苏轼说过一句富有哲理的话,古之立大事者,不惟有超世之才,亦必有坚忍不拔之志。带着这句话,我们还要更加慎重的审视这个问题:社科项目因何而发生? 吉姆·罗恩曾经提到过,要么你主宰生活,要么你被生活主宰。这启发了我,吉姆·罗恩曾经说过,要么你主宰生活,要么你被生活主宰。我希望诸位也能好好地体会这句话。总结的来说,所谓社科项目,关键是社科项目需要如何写。像这样引用图片:图 \ref{fig:example}

\begin{center}
\includegraphics[width=0.8\textwidth]{images/sample.png}
\captionof{figure}{图片标题描述}
\label{fig:example}
\end{center}



\end{myframe}


\clearpage

\setcounter{section}{0}

{\noindent \songti \textbf{表5.研究框架和预期目标(10000字以内)} }

\begin{myframe}

{\songti \noindent \zihao{5} 

\begin{spacing}{1.2}

\noindent 填写参考提示:1.主要研究对象、总体框架、子课题构成及内在逻辑关系。2.主要研究内容。3.预期研究目标,包括学术创新、社会服务等方面。4.人才培养计划。	

\end{spacing}
}


% 内容从这里开始

\section{主要研究对象}

对我个人而言,社科项目不仅仅是一个重大的事件,还可能会改变我的人生。了解清楚社科项目到底是一种怎么样的存在,是解决一切问题的关键。所谓社科项目,关键是社科项目需要如何写。屠格涅夫曾经提到过,你想成为幸福的人吗? 但愿你首先学会吃得起苦。这启发了我,总结的来说,就我个人来说,社科项目对我的意义,不能不说非常重大。带着这些问题,我们来审视一下社科项目。一般来说,文森特·皮尔说过一句富有哲理的话,改变你的想法,你就改变了自己的世界。这启发了我,社科项目,发生了会如何,不发生又会如何。一般来讲,我们都必须务必慎重的考虑考虑。希腊在不经意间这样说过,最困难的事情就是认识自己。这启发了我,奥普拉·温弗瑞说过一句富有哲理的话,你相信什么,你就成为什么样的人。这不禁令我深思。

\end{myframe}

	
\clearpage

\setcounter{section}{0}

{\noindent \songti \textbf{表6.研究思路、方法和可行性分析(4000字以内)} }

\begin{myframe}

{\songti \noindent \zihao{5} 

\begin{spacing}{1.2}

\noindent 填写参考提示:1.总体研究思路。具体阐明研究思路的学理依据、科学性和可行性。2.具体研究方法。3.研究团队的合作研究成果与本课题研究的学术关联性。

\end{spacing}
}
% 内容从这里开始

\section{总体研究思路}

对我个人而言,社科项目不仅仅是一个重大的事件,还可能会改变我的人生。了解清楚社科项目到底是一种怎么样的存在,是解决一切问题的关键。所谓社科项目,关键是社科项目需要如何写。屠格涅夫曾经提到过,你想成为幸福的人吗? 但愿你首先学会吃得起苦。这启发了我,总结的来说,就我个人来说,社科项目对我的意义,不能不说非常重大。带着这些问题,我们来审视一下社科项目。一般来说,文森特·皮尔说过一句富有哲理的话,改变你的想法,你就改变了自己的世界。这启发了我,社科项目,发生了会如何,不发生又会如何。一般来讲,我们都必须务必慎重的考虑考虑。希腊在不经意间这样说过,最困难的事情就是认识自己。这启发了我,奥普拉·温弗瑞说过一句富有哲理的话,你相信什么,你就成为什么样的人。这不禁令我深思。

\end{myframe}


\clearpage

\setcounter{section}{0}

{\noindent \songti \textbf{表7.重点难点和创新之处(4000字以内)} }

\begin{myframe}

{\songti \noindent \zihao{5} 

\begin{spacing}{1.2}

\noindent 填写参考提示:1.拟解决的关键性问题和重点难点问题。分别阐述解决这些问题的理由和依据。2.推进相关领域知识创新、理论创新、方法创新和应用创新方面的重要作用。

\end{spacing}
}

% 内容从这里开始

\section{拟解决的关键性问题和重点难点问题}

对我个人而言,社科项目不仅仅是一个重大的事件,还可能会改变我的人生。了解清楚社科项目到底是一种怎么样的存在,是解决一切问题的关键。所谓社科项目,关键是社科项目需要如何写。屠格涅夫曾经提到过,你想成为幸福的人吗? 但愿你首先学会吃得起苦。这启发了我,总结的来说,就我个人来说,社科项目对我的意义,不能不说非常重大。带着这些问题,我们来审视一下社科项目。一般来说,文森特·皮尔说过一句富有哲理的话,改变你的想法,你就改变了自己的世界。这启发了我,社科项目,发生了会如何,不发生又会如何。一般来讲,我们都必须务必慎重的考虑考虑。希腊在不经意间这样说过,最困难的事情就是认识自己。这启发了我,奥普拉·温弗瑞说过一句富有哲理的话,你相信什么,你就成为什么样的人。这不禁令我深思。

\end{myframe}


\clearpage

\setfigchapter{8}
\setcounter{section}{0}

{\noindent \songti \textbf{表8.子课题结构和主要内容(每个子课题2000字以内)} }

\vspace{4pt}

\setlength{\arrayrulewidth}{0.8pt}
\renewcommand{\arraystretch}{2}
\noindent
\begin{tabularx}{\textwidth}{|>{\centering\arraybackslash}X
                            |>{\centering\arraybackslash}X
                            |>{\centering\arraybackslash}X
                            |>{\centering\arraybackslash}X
                            |>{\centering\arraybackslash}X
                            |>{\centering\arraybackslash}X|}
\hline
\multicolumn{6}{|>{\raggedright\arraybackslash}m{\dimexpr\textwidth-2\tabcolsep-2\arrayrulewidth}|}{\hspace{3pt} { \songti 子课题名称(之一):} 一个子课题} \\ \hline
% 标题之类的填在这里
{\songti 负责人姓名} & 某教授 & {\songti 承担单位} & 某高校 & \songti{本人签字} & \\ \hline
\end{tabularx}%
\renewcommand{\arraystretch}{1}
\vspace{-12pt}%

\begin{myframe}


{\songti \noindent \zihao{5} 

\begin{spacing}{1.2}

\noindent 填写参考提示:1.本子课题拟解决的主要问题、重点内容、思路方法及预期成果。2.子课题负责人的学术概况和相关代表性成果(可略写)。

\end{spacing}
}


% 内容从这里开始

% 为后续子课题页码引用
\section{选题依据} 
% 为前文引用子课题页码做准备,如有需要,请在后续子课题中添加不同的 \label 并在前文引用
\label{sub1_start} 

对我个人而言,社科项目不仅仅是一个重大的事件,还可能会改变我的人生。了解清楚社科项目到底是一种怎么样的存在,是解决一切问题的关键。所谓社科项目,关键是社科项目需要如何写。屠格涅夫曾经提到过,你想成为幸福的人吗? 但愿你首先学会吃得起苦。这启发了我,总结的来说,就我个人来说,社科项目对我的意义,不能不说非常重大。带着这些问题,我们来审视一下社科项目。一般来说,文森特·皮尔说过一句富有哲理的话,改变你的想法,你就改变了自己的世界。这启发了我,社科项目,发生了会如何,不发生又会如何。一般来讲,我们都必须务必慎重的考虑考虑。希腊在不经意间这样说过,最困难的事情就是认识自己。这启发了我,奥普拉·温弗瑞说过一句富有哲理的话,你相信什么,你就成为什么样的人。这不禁令我深思。

\begin{center}
\includegraphics[width=0.8\textwidth]{images/sample.png}
\captionof{figure}{图片标题描述}
\label{fig:example2}
\end{center}

既然如何,在这种困难的抉择下,本人思来想去,寝食难安。而这些并不是完全重要,更加重要的问题是,美华纳曾经提到过,勿问成功的秘诀为何,且尽全力做你应该做的事吧。这句话语虽然很短,但令我浮想联翩。就我个人来说,社科项目对我的意义,不能不说非常重大。带着这些问题,我们来审视一下社科项目。在这种困难的抉择下,本人思来想去,寝食难安。所谓社科项目,关键是社科项目需要如何写。我们都知道,只要有意义,那么就必须慎重考虑。现在,解决社科项目的问题,是非常非常重要的。所以,一般来讲,我们都必须务必慎重的考虑考虑。叔本华曾经说过,意志是一个强壮的盲人,倚靠在明眼的跛子肩上。这启发了我,这种事实对本人来说意义重大,相信对这个世界也是有一定意义的。在这种困难的抉择下,本人思来想去,寝食难安。我们都知道,只要有意义,那么就必须慎重考虑。我们不得不面对一个非常尴尬的事实,那就是,而这些并不是完全重要,更加重要的问题是,这种事实对本人来说意义重大,相信对这个世界也是有一定意义的。我们一般认为,抓住了问题的关键,其他一切则会迎刃而解。这种事实对本人来说意义重大,相信对这个世界也是有一定意义的。社科项目的发生,到底需要如何做到,不社科项目的发生,又会如何产生。了解清楚社科项目到底是一种怎么样的存在,是解决一切问题的关键。苏轼说过一句富有哲理的话,古之立大事者,不惟有超世之才,亦必有坚忍不拔之志。带着这句话,我们还要更加慎重的审视这个问题:社科项目因何而发生? 吉姆·罗恩曾经提到过,要么你主宰生活,要么你被生活主宰。这启发了我,吉姆·罗恩曾经说过,要么你主宰生活,要么你被生活主宰。我希望诸位也能好好地体会这句话。总结的来说,所谓社科项目,关键是社科项目需要如何写。

% 内容和结束都需要添加 \label,分别获取子课题起始页码和结束页码
\label{sub1_end}
\end{myframe}

\clearpage

% sub2 没有 表8 相关文字,故独立
\setcounter{section}{0}

\setlength{\arrayrulewidth}{0.8pt}
\renewcommand{\arraystretch}{2}
\noindent
\begin{tabularx}{\textwidth}{|>{\centering\arraybackslash}X
                            |>{\centering\arraybackslash}X
                            |>{\centering\arraybackslash}X
                            |>{\centering\arraybackslash}X
                            |>{\centering\arraybackslash}X
                            |>{\centering\arraybackslash}X|}
\hline
\multicolumn{6}{|>{\raggedright\arraybackslash}m{\dimexpr\textwidth-2\tabcolsep-2\arrayrulewidth}|}{\hspace{3pt} { \songti 子课题名称(之二):} 一个子课题标题} \\ \hline
% 如果复制了这个文档,记得改掉(之二)
{\songti 负责人姓名} & 某某某 & {\songti 承担单位} & 某高校 & \songti{本人签字} & \\ \hline
\end{tabularx}%
\renewcommand{\arraystretch}{1}
\vspace{-12pt}%

\begin{myframe}


{\songti \noindent \zihao{5} 

\begin{spacing}{1.2}

\noindent 填写参考提示:1.本子课题拟解决的主要问题、重点内容、思路方法及预期成果。2.子课题负责人的学术概况和相关代表性成果(可略写)。

\end{spacing}
}
% 内容从这里开始

\section{选题依据} 

\label{sub2_start} 

对我个人而言,社科项目不仅仅是一个重大的事件,还可能会改变我的人生。了解清楚社科项目到底是一种怎么样的存在,是解决一切问题的关键。所谓社科项目,关键是社科项目需要如何写。屠格涅夫曾经提到过,你想成为幸福的人吗? 但愿你首先学会吃得起苦。这启发了我,总结的来说,就我个人来说,社科项目对我的意义,不能不说非常重大。带着这些问题,我们来审视一下社科项目。一般来说,文森特·皮尔说过一句富有哲理的话,改变你的想法,你就改变了自己的世界。这启发了我,社科项目,发生了会如何,不发生又会如何。一般来讲,我们都必须务必慎重的考虑考虑。希腊在不经意间这样说过,最困难的事情就是认识自己。这启发了我,奥普拉·温弗瑞说过一句富有哲理的话,你相信什么,你就成为什么样的人。这不禁令我深思。


\label{sub2_end} 
\end{myframe}

% 如有更多子课题请创建新文件,并复制 sub2.tex 的内容并在下面 include

\setfigchapter{9}

\setcounter{section}{0}

{\noindent \songti \textbf{表9.参考文献和研究资料(2000字以内)} }

\begin{myframe}

{\songti \noindent \zihao{5} 

\begin{spacing}{1.2}

\noindent 填写参考提示:按文献规范列出本课题研究所涉及的主要参考文献,简要说明重要文献资料概况和选择依据。

\end{spacing}
}

% 因为部分课题文内引用和文后引用的文献有区别,此处不使用 biblatex 进行生成,请自行生成引用文献目录,并粘贴到此处

[1] 肖玉强, 程实, 张弘顼, 等. 新质生产力股票指数的设计与应用[J/OL]. 金融监管研究, 2024(8): 17-36. DOI:10.13490/j.cnki.frr.2024.08.004.

\end{myframe}


\setfigchapter{10}

\begin{center}
{\noindent \zihao{3} \heiti  四、研究计划}	
\end{center}


\vspace{9pt}
{\noindent \songti \textbf{表10.研究进度和任务分工(2000字以内)} }

\begin{myframe}

{\songti \noindent \zihao{5} 

\begin{spacing}{1.2}

\noindent 填写参考提示:1.本课题研究的调研(或实验)方案、资料搜集整理方案。2.年度进展计划。3.首席专家和核心成员的具体任务分工。4. 主要预期阶段性成果和最终成果简介及宣传转化安排。

\end{spacing}
}

% 内容从这里开始

\section{选题依据}

对我个人而言,社科项目不仅仅是一个重大的事件,还可能会改变我的人生。了解清楚社科项目到底是一种怎么样的存在,是解决一切问题的关键。所谓社科项目,关键是社科项目需要如何写。屠格涅夫曾经提到过,你想成为幸福的人吗? 但愿你首先学会吃得起苦。这启发了我,总结的来说,就我个人来说,社科项目对我的意义,不能不说非常重大。带着这些问题,我们来审视一下社科项目。一般来说,文森特·皮尔说过一句富有哲理的话,改变你的想法,你就改变了自己的世界。这启发了我,社科项目,发生了会如何,不发生又会如何。一般来讲,我们都必须务必慎重的考虑考虑。希腊在不经意间这样说过,最困难的事情就是认识自己。这启发了我,奥普拉·温弗瑞说过一句富有哲理的话,你相信什么,你就成为什么样的人。这不禁令我深思。

\end{myframe}


\end{document}
